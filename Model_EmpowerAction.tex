\documentclass{article}
\usepackage{enumitem}
\usepackage{amsmath}
\begin{document}

\section*{EmpowerAction : Economia Descentralizada}

El documento presenta un marco integral para nuestra economía descentralizada "EmpowerAction." 
Se enfoca en la estructuración y regulación de precios por oferta y demanda, balances, emisiones, y circulación de tokens, así como en la gestión de créditos dentro del sistema.


\subsection*{1. Precio del Producto o Servicio}
\begin{itemize}
    \item El precio está definido por la oferta y la demanda y se cotiza en múltiples criptoactivos.
    \item Precio del producto o servicio en USD: \( P \)
\end{itemize}
\subsection*{2. Balance del Ecosistema}
\subsubsection*{a. Ingresos}
\begin{itemize}
    \item Comisiones por transacción (a definir): Las posibles comisiones están en el rango: \( A \in [0.002, 0.04] \). Aplicadas a las transacciones dentro del sistema.
    \item Comisión por niveles (A definir): Niveles de Lealtad: \( N: f(x1,\ldots,x4) \). Los niveles de lealtad pueden influir en las comisiones, ofreciendo incentivos para los usuarios frecuentes.
    \item Comisión por transacciones \( C_m = P \times A \times N \). Comisión  basada en el precio y niveles de lealtad.
    \item Comisión por transacciones abonadas con EMP(Token nativo): \( C_{me} = P \times A \times N \times 0.5 \). 
    \item Ingresos totales en USD: \( \text{ING} = \sum C_m + C_{me} \). Suma de todas las comisiones para calcular los ingresos totales.
\end{itemize}
\subsubsection*{b. Egresos}

\begin{itemize}
    \item \( M_f \in [0.001, 0.48] \): Factor que representa el margen de gastos.
    \item Gastos USD de Mantenimiento \( G_m \) (De mayor a menor, con objetivo de en 1\%): \( G_m = \text{ING} \times M_f \). Calcula los gastos de mantenimiento basados en los ingresos y el margen.
    \item Gastos en USD Asignados a la inyección de liquidez: $R_yq$ = ${ING}$ - $G_m$. Esto determina la cantidad de dinero destinada a aumentar la liquidez en el sistema.
    \item EMP: Token nativo. VALOR DOLAR: \( \text{EMP}_v \). Define el valor en dólares del token nativo.
    \item Cantidad de EMP a recomprar y quemar $(EM)_q$: ${EM}_q$ = $R_yq$ / ${EMP}_v$ . Calcula la cantidad de tokens a recomprar y eliminar.
    \item Si \( R_yq \) Absorbe la oferta total de tokens, el resto se convierte en \( R \). Esta condición asegura que la oferta y la demanda de tokens estén equilibradas.
    \item Reservas en USD (R): \( R = R_yq - (\text{EM}_q \times \text{EMP}_v) \).
    \item \( R \) es utilizado para vigorizar la liquidez a través de diversos mecanismos decididos en consenso y basados en datos. Esto permite que el sistema mantenga una operación fluida y eficiente.
\end{itemize}



\subsection*{3. Emisión y Circulación de Tokens basada en créditos (EMP y DEUS)}
El Crédito se emite cuando un usuario tiene fondos insuficientes para abonar servicios de una denominación menor al 40\% de su crédito en USD total en la plataforma, Este se liberará directamente en la cuenta del proveedor. 

\subsubsection*{a. Emisión por Crédito Utilizado}
\subsubsection*{Créditos Mutuos}
\begin{align*}
    \text{CR}_d & \in \{x1,\ldots,x150\}, \text{ donde } x \text{ Representa el rango de créditos disponibles.} \\
    \text{Crédito Total en USD (CRDt):} & \text{ Vinculado a las Reservas, Crédito total disponible en el sistema.} \\
    \text{CRusuario} & = \text{CRDu} \times \text{ factor de comportamiento, podría estar en el rango } [0.5, 1.5] 
\end{align*}

\subsubsection*{EMP}
Se emite la cantidad de tokens representativa por valor nominal del servicio contratado (Definido entre partes). Esto vincula los tokens con el valor real de los servicios.

\subsubsection*{Articulación}
\begin{align*}
    E_i & = \text{Cantidad inicial de tokens en circulación.} \\
    \text{Emisión de deuda ejecutada por Crédito Utilizado: CR}_e & = P. \text{ Basada en el precio del servicio.} \\
    \text{Emisión de token EMP x Credito utilizado: EMP}_e & = \text{EMP}_v \times \text{CR}_e\\
    \text{Emisión total EMP: EMP}_et & = E_{\text{inicial}} + \sum \text{ EMP}_e - \text{EMP}_q. \\
\end{align*}

El prestatario recibirá “DUS” wrapped intransferible y eliminable por parte de la misma plataforma. Instrumento de Deuda.

\subsubsection*{DUS}

El Pago se realiza en EMP o USD o ETH, libera su cupo de deuda DUS y permite Eliminar DUS de su wallet. Permitiendo utilizar nuevamente la plataforma. Esto proporciona un mecanismo para gestionar y liquidar deudas dentro del sistema.

\subsubsection*{Articulación}
\begin{align*}
    \text{Crédito total en USD inicialmente (CD):} &\ 10 \text{ Dólar por cuenta, basado en las reservas.} \\
    \text{Valor de Wrapped DUS:} &\ 1 \text{ Dólar} \\
    \text{Emisión Total DUS: DUS}_e & = \text{DUS}_v \times \text{CR}_e. \text{ Emisión de deuda.} \\
    \text{Emisión total deuda a usuario (DUS)}_u: &\ \text{DUS}_u = \sum \text{DUS}_e < D. \\
    \text{Tokens DUS a Quemar: DES}_q & = \text{D}pago. \text{ Define los tokens a eliminar.} \\
    \text{Emisión total de deuda en dólares de la plataforma (DUS)}_t: &\ \text{DUS}_t = \sum \text{DUS}_u
\end{align*}


\subsubsection*{4. Política de Recomprar y Quemar}
\begin{itemize}
    \item Tokens Quemados \(EMq = \frac{Ryq}{EMP_v}\)
    \item Si \(R_yq\) Absorbe la oferta Total de Tokens:
    \item Si \(EMP_q > \text{EMPtotal}\), entonces \(R = R + R_yq - (EMP_q \times EMP_v)\).
    \item Si \(EMP_q < \text{EMPtotal}\), entonces \(R = R + 0\).
    \item Actualización de la oferta de tokens:
    \item \(EMPtotal\_nuevo = EMPtotal - EMP_q\). Actualiza la oferta total de tokens en circulación.
\end{itemize}


\subsubsection*{En resumen:}
EmpowerAction propone un sistema económico descentralizado robusto y flexible, con mecanismos claros y transparentes para la gestión de precios, comisiones, tokens, y créditos. La estructura presentada busca ser una base para una operación fluida y eficiente, incentivando la participación, la transparencia y la responsabilidada indiavidual, permitiendo la adaptabilidad necesaria a las condiciones cambiantes del mercado. 


\end{document}
